\documentclass{moderncv}

\moderncvstyle{fancy}
\moderncvcolor{blue}

\usepackage[utf8]{inputenc}
\usepackage[polish]{babel}

\usepackage[left=0.3in, right=1in, top=1in, bottom=1in]{geometry}
\setlength{\hintscolumnwidth}{4.78cm}
\recomputelengths

\name{Krzysztof}{Gutkowski}
\title{Fullstack Developer}
\phone[mobile]{+48 500 535 818}
\email{krzysio.gutkowski@gmail.com}
\social[github]{LiquidPL}

\begin{document}

\makecvtitle

\section{Podsumowanie kariery}
\cvitem{}{Jestem Fullstack Developerem z 9 latami doświadczenia w języku PHP, oraz 5 latami doświadczenia w JavaScript/TypeScript. Moją główną specjalizacją po stronie backendu jest Laravel oraz Symfony, a w przypadku frontendu React. Rozwijam również swoje umiejętności we frameworkach Vue oraz Angular. Posiadam doświadczenie w pracy w dużych projektach, zarówno nowych, jak i legacy.}

\section{Doświadczenie zawodowe}
\cventry{2020/10 - 2024/08}{programista backend PHP}{Macopedia}{zdalnie/Poznań}{}{
    \begin{itemize}
        \item rozwój projektów dla klientów firmy:
        \begin{itemize}
            \item system zarządzania treścią (TYPO3/Symfony)
            \item autorski portal eCommerce B2B (Symfony)
        \end{itemize}
        \item implementacja wtyczek oraz wysyłanie poprawek do upstreamowego projektu TYPO3.
    \end{itemize}
}
\cventry{2019/11 - 2020/03}{programista PHP/full stack}{mediacyfrowe.pl - doradztwo, realizacja}{Warszawa}{}{
    \begin{itemize}
        \item rozwój projektów oraz wykonywanie zleceń dla klientów firmy,
        \item tworzenie autorskich projektów dla firmy (w tym przypadku były to wtyczki do znanych systemów typu \textit{self-hosted}, np. WordPress, PrestaShop),
        \item obsługa wydarzeń organizowanych przez klientów (np. obsługa skanowania biletów podczas wydarzenia do którego przez firmę został stworzony system sprzedaży).
    \end{itemize}}
\cventry{2016 - 2019}{programista Laravel/React}{osu! (\textit{\href{https://osu.ppy.sh}{https://osu.ppy.sh}})}{zdalnie}{}{
    Przebudowa przestarzałej wersji strony (zbudowanej na bazie \textit{phpBB}) oraz backendu/API gry sieciowej (ok. 1 miliona użytkowników miesięcznie) z wykorzystaniem stacku Laravel oraz React.
    \begin{itemize}
        \item adaptacja istniejących tabel bazy danych na modele ORM,
        \item implementacja komponentów/podstron w React oraz odpowiadających im endpointów API,
        \item zachowanie kompatybilności z istniejącą wersją strony (w trakcie prac obie strony działały jednocześnie, na tej samej produkcyjnej bazie danych).
    \end{itemize}
}
\cventry{2019 - teraz}{administrator sieci/programista}{cavoe's osu! event (COE)}{online/Den Bosch, Holandia}{wolontariat}{
    Wsparcie techniczne oraz programistyczne dla imprezy masowej (ok. 1000 uczestników przez 7 dni) skupionej na grach komputerowych.
    \begin{itemize}
        \item planowanie oraz wdrażanie fizycznej strukturą sieci oraz konfiguracja routerów/switchy/access pointów/itp.,
        \item implementacja grafik oraz paneli wykorzystywane podczas transmisji na żywo (Vue 3),
        \item rozwój strony internetowej z informacjami o imprezie oraz systemem sprzedaży biletów (NestJS + Angular).
    \end{itemize}
}
\cventry{2015 - teraz}{administrator serwerów Linux}{VI LO im. Jana Kochanowskiego}{Radom}{}{Administracja i konserwacja systemu wykorzystywanego do oceny zadań i przeprowadzania zajęć z informatyki oraz obozów informatycznych prowadzonych przez szkołę.}

\section{Wykształcenie}
\cventry{2020/10 - 2025/10}{inżynier}{Wydział Elektryczny Politechniki Warszawskiej}{Warszawa}{}{Informatyka Stosowana}

\section{Umiejętności}
\cvitem{Języki programowania}{
    \begin{itemize}
        \item PHP - b. zaawansowany (ok. 9 lat doświadczenia),
        \item JavaScript/TypeScript - zaawansowany (ok. 5 lat doświadczenia)
        \item Python - podstawowy
        \item C++ - podstawowy
        \item Rust - początkujący
    \end{itemize}
}
\cvitem{Frameworki backendowe}{
    \begin{itemize}
        \item Laravel - zaawansowany (ok. 4 lata doświadczenia)
        \item Symfony - zawansowany (ok. 5 lat doświadczenia)
    \end{itemize}
}
\cvitem{Technologie frontendowe}{
    \begin{itemize}
        \item React - zaawansowany (ok. 4 lat doświadczenia)
        \item Vue, Angular - początkujący
        \item Less/SCSS
    \end{itemize}
}
\cvitem{Inne}{Git, Docker, Kubernetes, zaawansowana znajomość systemu Linux.}
\cvlanguage{Język angielski}{CEFR C1}{Cambridge Certificate in Advanced English}

\section{Zainteresowania}
\cvlistdoubleitem{Książki (fantasy, science fiction)}{Jazda na rowerze}
\cvlistdoubleitem{Pływanie}{Narciarstwo}
\cvlistitem{Gry komputerowe}

\vfill
\begin{center}
\textit{\small Wyrażam zgodę na przetwarzanie moich danych osobowych dla potrzeb niezbędnych do realizacji procesu tej oraz przyszłych rekrutacji (zgodnie z ustawą z dnia 10 maja 2018 roku o ochronie danych osobowych (Dz. Ustaw z 2018, poz. 1000) oraz zgodnie z Rozporządzeniem Parlamentu Europejskiego i Rady (UE) 2016/679 z dnia 27 kwietnia 2016 r. w sprawie ochrony osób fizycznych w związku z przetwarzaniem danych osobowych i w sprawie swobodnego przepływu takich danych oraz uchylenia dyrektywy 95/46/WE (RODO)).}
\end{center}

\end{document}
