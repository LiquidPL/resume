\documentclass{moderncv}

\moderncvstyle{fancy}
\moderncvcolor{blue}

\usepackage[utf8]{inputenc}
\usepackage[polish]{babel}

\usepackage[left=0.3in, right=1in, top=1in, bottom=1in]{geometry}
\setlength{\hintscolumnwidth}{4.78cm}
\recomputelengths

\name{Krzysztof}{Gutkowski}
\title{Full-Stack Developer}
\phone[mobile]{500535818}
\email{krzysio.gutkowski@gmail.com}
\address{Adres zameldowania: \\Cerekiew (pod Radomiem),\\ ul. Radomska 149\\ 26-652 Zakrzew\\ Adres zamieszkania: \\ul. Księcia Janusza 39 \\01-452 Warszawa}
\social[github]{LiquidPL}

\begin{document}

\makecvtitle

\section{Doświadczenie zawodowe}
\cventry{2019/11-2020/03}{programista PHP/full stack}{mediacyfrowe.pl - doradztwo, realizacja}{Warszawa}{}{
    Praca opierała się głównie na wykonywaniu zleceń dla klientów firmy.
    \begin{itemize}
        \item konserwacja i dalszy rozwój już istniejących projektów,
        \item planowanie oraz tworzenie nowych projektów,
        \item tworzenie autorskich projektów dla firmy (w tym przypadku były to wtyczki do znanych systemów typu \textit{self-hosted}, np. WordPress, PrestaShop),
        \item obsługa wydarzeń organizowanych przez klientów (np. obsługa skanowania biletów podczas wydarzenia do którego przez firmę został stworzony system sprzedaży).
    \end{itemize}}
\cventry{2016--2019}{programista Laravel/React}{osu! (\textit{\href{https://osu.ppy.sh}{https://osu.ppy.sh}})}{zdalnie}{}{Rozwój strony (backend oraz frontend) dla średnich rozmiarów gry online (1 milion użytkowników miesięcznie).}
\cventry{2015--teraz}{administrator serwerów Linux}{VI LO im. Jana Kochanowskiego}{Radom}{}{Administracja i konserwacja systemu wykorzystywanego do oceny zadań i przeprowadzania zajęć z informatyki oraz obozów informatycznych prowadzonych przez szkołę.}

\section{Wykształcenie}
\cventry{2016--2019, 2020--teraz}{student}{Wydział Elektryczny Politechniki Warszawskiej}{Warszawa}{studia zaoczne}{}
\cventry{2013--2016}{uczeń liceum}{VI LO im. Jana Kochanowskiego}{Radom}{}{}
\cventry{2010--2013}{uczeń gimazjum}{PG nr 23 im. Jana Kochanowskiego}{Radom}{}{}

\section{Języki obce}
\cvlanguage{angielski}{CEFR C1}{Cambridge Certificate in Advanced English}

\section{Umiejętności}
\cvitem{Języki programowania}{PHP, JavaScript/TypeScript, Python.}
\cvitem{Bazy danych}{MySQL, PostgreSQL.}
\cvitem{Frameworki}{Laravel, Symfony, Django.}
\cvitem{Technologie frontendowe}{HTML5/CSS3, React, Vue, Less.}
\cvitem{Inne}{Git, Docker, zaawansowana znajomość systemu Linux.}

\section{Zainteresowania}
\cvlistdoubleitem{Książki (fantasy, science fiction)}{Jazda na rowerze}
\cvlistdoubleitem{Pływanie}{Narciarstwo}
\cvlistitem{Gry komputerowe}

\section{}\closesection
\cvitem{}{Wyrażam zgodę na przetwarzanie moich danych osobowych dla potrzeb niezbędnych do realizacji procesu tej oraz przyszłych rekrutacji (zgodnie z ustawą z dnia 10 maja 2018 roku o ochronie danych osobowych (Dz. Ustaw z 2018, poz. 1000) oraz zgodnie z Rozporządzeniem Parlamentu Europejskiego i Rady (UE) 2016/679 z dnia 27 kwietnia 2016 r. w sprawie ochrony osób fizycznych w związku z przetwarzaniem danych osobowych i w sprawie swobodnego przepływu takich danych oraz uchylenia dyrektywy 95/46/WE (RODO)).}

\end{document}
